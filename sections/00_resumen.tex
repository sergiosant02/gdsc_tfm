\chapter*{Resumen}

Las terapias personalizadas genéticamente para tratar el cáncer muestran grandes avances, permitiendo mejorar la calidad y esperanza de vida de los pacientes. Sin embargo, existe un reto clave en este tipo de tratamientos, ya que actualmente no se puede predecir con exactitud la respuesta de una persona ante un medicamento, lo que supone un gran problema. Si el tratamiento es demasiado intenso, no solo no ayudará al paciente, sino que supondrá un empeoramiento de su condición.

Este estudio hace uso del dataset \textit{Genomics of Drug Sensitivity in Cancer (GDSC)}, el cual incluye información sobre perfiles genómicos, farmacológicos y datos sobre las características de las líneas celulares cancerosas, constituyendo una gran fuente de información. Para explotar estos datos, se han empleado algoritmos basados tanto en redes neuronales como en árboles, manteniendo siempre como objetivo proporcionar un modelo que no solo sea preciso, sino útil para la sociedad.

La investigación trata de obtener un modelo que sea capaz de predecir la variable LNIC50, la cual es un indicador de la concentración necesaria de medicamento para inhibir el crecimiento de las células cancerosas en un 50%.

El modelo resultante ha sido, a su vez, validado mediante técnicas de explicabilidad, proporcionando certeza sobre la fiabilidad de las predicciones y demostrando que los perfiles genéticos influyen en la respuesta del paciente ante los fármacos contra el cáncer.

\vspace{.5cm}

\textbf{Palabras clave:} cáncer, respuesta farmacológica, efectividad del tratamiento, redes neuronales, XGBoost, explicabilidad, SHAP.