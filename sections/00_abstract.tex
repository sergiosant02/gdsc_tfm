\chapter*{Abstract}
Genetically personalised therapies for treating cancer are showing great advances, improving patients' quality of life and life expectancy. However, there is a key challenge in this type of treatment, as it is currently impossible to accurately predict a person's response to a drug, which is a major problem. If the treatment is too intense, not only will it not help the patient, but it will also worsen their condition.

This study makes use of the Genomics of Drug Sensitivity in Cancer (GDSC) dataset, which includes information on genomic and pharmacological profiles and data on the characteristics of cancer cell lines, constituting a great source of information. To exploit this data, algorithms based on both neural networks and trees have been used, always with the aim of providing a model that is not only accurate but also useful to society.

The research aims to obtain a model capable of predicting the  \(LN\_IC_{50}\) variable, which is an indicator of the concentration of drug needed to inhibit the growth of cancer cells by 50%.

The resulting model has, in turn, been validated using explainability techniques, providing certainty about the reliability of the predictions and demonstrating that genetic profiles influence the patient's response to cancer drugs.

\vspace{.5cm}

\textbf{Keywords:} cancer, drug response, treatment effectiveness, neural networks, XGBoost, explainability, SHAP.