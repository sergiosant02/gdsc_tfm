\chapter{Introduction and motivations}\label{cap:introduccion}

There is currently an extensive body of publicly available research foccused on finding a cure for cancer. There is a huge amount of open access data \cite{tcga_nci, gdsc_website} that allows researchers worldwide to exchange knowledge with the community. This is how the scientific community tries to continue the chain until it achieves findings that can save lifes. 

One line of research is the effect of a drug based on the patient's genome. There are records of patients and studies with the same medication and type of cancer but with different responses to treatment \cite{genomicBiology, metastasisCancer}. Especially regarding the personalization of therapy since the variability raises critical questions. Decisions about which medication to administer and in what dosage directly affect people's quality of life. If a patient receives an excessive amount of medication, the consequences can be even worse than those of the cancer itself. Similarly, if the amount administered is less than required, the effects will not be as desired.

One of the main challenges of applying machine learning techniques, to this type of research, lies in the very nature of the data, which poses significant limitations. Generally, it is highly variable, with cancer records showing a large number of treatments for the same type and often containing a considerable amount of noise or missing values. At the same time, genetic information is very extensive, further complicating the search for a model that meets all requirements.

To address these challenges, the proposed models must be highly robust in the face of different datasets. They must be able to function correctly despite noise and missing values. In addition, these models must be updated regularly so that new knowledge about the disease leads to better and higher quality predictions. Without these periodic predictions, the risk of obsolescence increases dangerously.

This is undoubtedly one of the most demanding and sensitive areas of research, not only due to its technical complexity, but also because of the profound human responsibility it entails. Working with data derived from serious illnesses, such as cancer, requires a constant awareness that behind each data point is a person, with their own story, struggle and hope. The outcomes of such research have the potential to significantly improve patient care and quality of life. However, this potential is accompanied by a high degree of responsibility: a robust and well-validated finding can guide more effective treatments, while an error or oversight could lead to inappropriate clinical decisions, potentially harming lives. In this respect, it really is a matter of life or death. It is a field that demands not only technical excellence but also ethical rigor, caution and empathy. The risks are substantial but so are the opportunities to make meaningful, life-changing contributions to human health.